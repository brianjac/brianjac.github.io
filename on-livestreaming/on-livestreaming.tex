\documentclass{article}
\usepackage{enumerate}
\usepackage{hyperref}
\usepackage[threshold=1]{csquotes}

\fboxsep=20pt


\title{On Livestreaming}
\date{August 20th, 2018}

\begin{document}
\maketitle
\begin{center}
  \href{../index.html}{All Articles}
\end{center}

I've been thinking a lot about the business of twitch livestreaming a lot recently. I have friends who are involved with it (\href{https://www.twitch.tv/liealgebracow}{Lie Algebra Cow}, \href{https://www.twitch.tv/sugarfree007}{Sugarfree}, \href{https://www.twitch.tv/yoman5}{Yoman}) although I confess that I don't watch many streams myself. My personal experience is mostly limited to Rocket League Esports streams of varying descriptions and VODs of speedruns, races, and Rocket League showmatches. Except for live esports, my personal relationship with livestreaming is asynchronous, which perhaps defeats the point of streams being live in the first place.\\

This fundamental difference between produced content and live content seems like it introduces a number of problems. First and foremost, in order for a streamer to be successful, they have to be streaming at times that their audiences can watch. This means that a streamer has to stream for a long period of time to cover a wide range of timezones and availability if they want to target a general audience. Streaming for a long period of time affects the kind of content that they can produce. Anything that requires postproduction is out. No cutting out boring parts, or providing clarification when something confusing or sophisticated happens. Everything must be narrated in real time, and at least somewhat interesting to watch as well. If there's a boring segment of a stream, there are plenty of other streams which might be doing something more interesting that casual viewers could go watch instead. So we have a pressure to produce large amounts of content, in real time, which has a minimum level of engagement for the audience. An easy to see effect of this pressure is the idea of ``streaming goals'' where streamers attempt to keep a streak of consecutive days streaming, or reward viewers with 24 hour (or more) long streams.\\ 

The hook for a stream on Twitch is the game being played. Twitch separates streams into categories based on game, and has a cultural focus on gameplay as the content being produced. So a streamer needs to either be playing a popular game with enough of an external relationship to the game to bring in viewers, or play a variety of games to entice viewers who's discovery mechanism is an interest in watching a particular game be played.\\

This set of constraints favours streamers who are interesting on their own terms, not just because of the games that they are playing. Someone who has a strong and distinctive personality can carry viewers from game to game, exposing themselves to new pools of potential viewers without compromising their existing viewerbase. This also somewhat insulates them from the risk of having something related to the game compromise the nature of their stream. If they are the product, not the gameplay, then a bad day of content in a game---or a bad game entirely---won't entirely sink their stream. So in order to be successful livestreaming, you essentially need to be entertaining for at least most of the stream on your own, using the game you are playing as at most a foil.\\

When a streamer needs to both be interesting as a personality and stream for long periods is where I see things starting to go wrong. Even the most naturally charismatic and engaging people need some kind of downtime. They have off days and bad moods. So they adopt a kind of stage persona, a character who has an outsized and simplified personality allowing them to play a role instead of relying on being engaging as their unadulterated selves. They have an incentive to keep as close as possible to their real selves, so that it is easier to keep up the performance for hours on end, day after day. The other incentive, though, is to always be more and more entertaining, more and more interesting, more and more engaging.\\

An interesting example of this is Jace Connors. I first started watching Jace's livestreams in 2014. He would play a variety of games, drink, do drugs, and rant at length about conspiracy theories and how much he hated his deeply Christian mother. Oftentimes his streams were emotionally intense. One in particular I remember vividly, where he was playing The SIMs with characters representing himself and his parents. He had an emotional, sobbing breakdown for about half an hour as he symbolically caused his "dad" SIM to have a cage match with a John Cena SIM, who he intimated was the father he wished he had. Afterwards, he directed his SIM to kill his "dad" SIM and cut the broadcast, an absolute wreck. I sat quietly in the dark for a while after that stream ended, thinking about what had just happened.\\

As his streams gained traction, he was kicked off of several streaming platforms for doing drugs and generally being outside of the norms of those platforms, but he refined his technique. He had a call-in show where people would skype him and talk about conspiracy theories. Oftentimes people would send him surprise live video of their penises during these skype calls. This would reliably produce a meltdown where he vehemently denied being gay. So of course, people did it every chance they got.\\

At his peak, he believed that the rapper Tupac was alive and being held in Palestine by Hamas. He sent one of his friends to Israel to attempt to sneak across the border and rescue Tupac. They released videos of this friend both on the plane to Tel Aviv, and later wandering around the city. He posted a clip from a bus towards the border, and then went dark for several months.\\

At some point around this time he got into some kind of disagreement with Brianna Wu. It was most likely in response to Gamergate, which was very much in the public consciousness at the time. He started ranting about how he was going to "assassinate her via street racing." It might have been brought on by the fact that she rode a motorcycle, or it might have just been Jace's obsession with his status as a "street racing god."\\

The beginning of the end for Jace was when he posted a video of himself next to his flipped car by the side of the highway. He asserted that he had been street racing to try to get to Brianna Wu so that he could street race her, but she had attempted to have him assassinated. Clearly a street racing god such as himself would never have crashed unless someone had sabotaged him.\\

Eventually the legal ramifications of his street racing video forced him to reveal himself as a performance artist named SAMPLE NAME. "Jace Connors" was just a character he played. But people that lived near him, and people that played online games he joined asserted that he never broke character. He was devoted to being Jace Connors, and he wasn't going to risk someone finding out that Jace wasn't real. His character evolved with his audience, feeding into each other and building up until it hit a crescendo that he could no longer maintain. He crashed, and faded into obscurity.\\

When someone has to play a caricature of yourself for so long, where do they end and the caricature begins? The risk of doing something offensive or inappropriate or out of bounds is high. They're trying to explore the boundaries of what they've done before, attempting to find the direction that they can push in to increase their engagement numbers. There's a very large search space, and a basically immediate feedback loop. That's a recipe for addiction. You can see immediately how many people are watching you. If you're interesting, new viewers will follow or subscribe to you. People might donate money to you if you do something particularly exciting. Any time you experiment with something you get immediate feedback on whether it was successful or not, and that feedback is in the form of income.\\

These simplified characters are more engaging that more nuanced, real personalities, which makes them ideally engaging for younger people. As young teenagers and preteens are exploring who they are and learning how to form mature social relationships they are consuming stream media that shows these simplified, outsized personalities in a way that is very difficult to distinguish from non-performing people. The livestreamer cries out: "This is who I am." Their viewers echo: "This is who we want to be."\\

The call and response between the streamer and their viewers is a feedback loop. Once you have established yourself, like Jace did, you have to keep going. You have to continuously innovate and produce new content that tops that which you had produced before. This pushes your personality farther and farther. You have to do this for eight or more hours a day, every day. This sounds to me like some kind of hell, trapped in a continuous performance at the whim of your fickle audience.\\

The audience is not necessarily getting any kind of long-term good out of this relationship either. Watching a livestream strikes me as a potential replacement for other kinds of social interaction. If you don't have enough real world interactions with your peers, you can feel like a livestreamer is one of those peers. They talk regularly to their viewers, interact with them. They might not interact individually with a particular viewer, but that person is part of a community---the community that watches and engages with the streamer. Just as the streamer is molded playing a character, the audience who are participating and watching are molded by the personality of the character of the streamer.\\

If a streamer holds some kind of opinion, or does a particular thing, the audience emulates it. It's an opportunity to engage with the streamer, to be influenced by them. This means that the streamer has an enormous amount of power. They can influence all of the people watching them, especially the younger ones. But at the same time, they have almost no power at all. If they do or say something that their audience finds objectionable, then that audience will leave. Someone else will be willing to play a character that's more agreeable. The viewers cry out: "This is who we are." The livestreamer echos: "This is who I will become."\\

In this way the existing biases and traits of the audience are reinforced by their membership. The streamer can try to influence these biases, but at the cost of immediate negative feedback, in the form of lost viewership, lost followership, lost subscriptions, and lost donations. Taking a stand is something that hurts, both in the immediate negative feedback and the lost income over time. So the streamer is pressured not to say anything, not to do anything which might endanger their position.\\

This leads into the controversy which originally prompted this discussion: the streamer Ninja's announcement that he would not play on-stream with women, for fear of accusations of cheating on his wife. Facially, there's a pretty big problem with this. Women are a minority in video gaming culture. Most people who play the kinds of video games that get streamed on Twitch are male, and most of the viewers are as well. Any woman who is engaging with that culture is already an outsider, and many video gaming communities which already have issues with toxicity can be absolutely horrific to outsiders. Shunning women as ``other'' on a massive stage teaches that shunning women is correct, acceptable behaviour. Communities thrive when they are open and accepting of new ideas and people. At a broader scale, teaching young men to engage with women as people in the communities where they spend their time now will encourage them to engage with women as people in the rest of their lives.\\

Another, more immediate issue is that one of the most valuable currancies in livestreaming is exposure to a new market. We looked at this before when we were talking about the importance of streaming long hours to catch people across timezones and lifestyles. Exposure to new markets is how you grow as a streamer. One great way to do that is by cross-pollenation with someone else's audience. If you collaborate with someone you are introduced to their viewers, and they are introduced to yours. Those viewers more likely to watch your stream when they have the opportunity to do so in the future. This means that the people at the top of the industry have an enormous amount of power to boost those below them into success.\\

In a more traditional business setting, there are checks and balances. A manager cannot just promote or hire someone: Human Resources needs to approve of the action. Decisions made by a CEO are checked by the Board of Directors, decisions made by Marketing are checked by Legal, and the entire business is checked by external auditing and the shareholders. However in livestreaming oftentimes the streamer is the only one making decisions. There's no structure in place to provide guidance on navigating unusual scenarios. There's no accountability to anyone other than yourself. If Twitch had a policy of not showing livestreams of or to women there would be no question that the legal system would be brought to bear. But lots of people don't think of a streamer as a business. They're an individual, and so their business decisions are judged on an individual scale.\\

Just to be wholly unambiguous: despite the pressures acting upon someone in Ninja's position, deciding to solve the problem by ignoring it---by not working with any of the women in your profession---is wrong. Just because doing the right thing is hard does not mean you should not do it.\\

How might the livestreaming ecosystem recover? Large livestreamers need to appeal to their viewers, so providing support for those that take a stand against their audience's base desires is important. Showing that there is disapproval when a large streamer gives in to the demands of a vocal plurality of their viewers is also important, but there's a limit to how much it is reasonable to ask of those who are already somewhat outside of the community to work in order to get in. You cannot boycott if you are not involved in the first place.\\

So perhaps the issue that needs to be resolved is that the entire ecosystem needs to be restructured in such a way that taking risks is possible. There needs to be some sort of insulation between a streamer and the immediate feedback of their actions affecting their livelihood. Being able to take an evening off or make a controversial decision without immediately feeling like your career is at risk allows people to do things because they are right, not because the mob demands them. Hopefully, over time the ability to push back against the audience allows a streamer to teach them how to behave.\\

Beyond the vague goal of reducing the volatility of livestreaming as a profession and hoping that that naturally allows for moderation and growth in the viewer base, I really don't have any ideas for improving this misshappen, upside-down, mob-ruled landscape. I'd love to hear what other people think. Feel free to to me on Twitter: the thread for this article is \href{https://twitter.com/KleeneAlgebraCo/status/1031695417145286657}{here}.

\end{document}
